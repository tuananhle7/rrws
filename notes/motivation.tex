
\section{Objective - Motivation}
\label{sec:mot}

Why co-ordinate descent?
%
Consider first the expressions for the lower and upper bounds of the
log-marginal likelihood:
%
\begin{align*}
  \log \p[\theta]{y}
  &=\fnP{\ELBO}{\theta, \phi; y} + \KL{\q[\phi]{x \given y}}{\p[\theta]{x \given y}}\\
  \log \p[\theta]{y}
  &=\fnP{\EUBO}{\theta, \phi; y} - \KL{\p[\theta]{x \given y}}{\q[\phi]{x \given y}}\\
  \intertext{Where the bounds \(\ELBO\) and \(\EUBO\) are given as}
  \fnP{\ELBO}{\theta, \phi; y}
  &=\Ex[{\q[\phi]{x \given y}}]{\log \p[\theta]{x, y} - \log \q[\phi]{x \given y}}\\
  \fnP{\EUBO}{\theta, \phi; y}
  &=\Ex[{\p[\theta]{x \given y}}]{\log \p[\theta]{x, y} - \log \q[\phi]{x \given y}}
\end{align*}
%
The objective with these expressions is to learn the generative model
parameters~\(\theta\) that maximise the log-marginal likelihood or evidence.
%
\[\argmax_{\theta} \log \p[\theta]{y} \]

\todo[inline]{Transcribe remainder of objective from whiteboard sessions}

\todo[inline]{What is the right objective to pursue: \(U - L\) or \(U \text{and } L\) alternating?}

%%% Local Variables:
%%% mode: latex
%%% TeX-master: "notes"
%%% End:
